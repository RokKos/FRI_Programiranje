\documentclass{article}

\usepackage{hyperref}

\title{My First Document}
\date{15.05.2017}
\author{Bor Brecelj}

\begin{document}
	
	\section{Naloge}
		\paragraph{Navodila} Tukaj je se nekaj dodatnih nalog. Predlagam pa, da se najprej resi tutorske naloge iz spletne ucilnice. Naloge se odda na spletno stran in takoj dobis rezultat. Na te spletne strani se je potrebno registrirati ampak se splaca.

		\subsection{Tabele}
			\begin{itemize}
				\item \url{https://www.codingame.com/training/medium/stock-exchange-losses}
				\item \url{https://www.hackerrank.com/challenges/arrays-ds}
				\item \url{https://www.hackerrank.com/challenges/array-left-rotation}
				\item \url{https://www.hackerrank.com/challenges/2d-array}
			\end{itemize}

		\subsection{Nizi}
			\begin{itemize}
				\item \url{https://www.codingame.com/training/easy/ascii-art}
				\item \url{https://www.hackerrank.com/challenges/camelcase}
				\item \url{https://www.hackerrank.com/challenges/caesar-cipher-1}
				\item \url{https://www.hackerrank.com/challenges/mars-exploration}
				\item \url{https://www.hackerrank.com/challenges/funny-string}
				\item \url{https://www.hackerrank.com/challenges/pangrams}
			\end{itemize}

		\subsection{Sortiranje}
			\begin{itemize}
				\item \url{https://www.hackerrank.com/challenges/tutorial-intro} - insertion sort smo delali(to je urejanje z navadnim vstavljanjem)
				\item \url{https://www.hackerrank.com/challenges/closest-numbers}
			\end{itemize}

		\subsection{Rekurzija}
			\begin{itemize}
				\item \url{https://www.hackerrank.com/challenges/recursive-digit-sum}
			\end{itemize}

\end{document}